%%%%%%%%%%%%%%%%%
% This is an sample CV template created using altacv.cls
% (v1.6.5, 3 Nov 2022) written by LianTze Lim (liantze@gmail.com). Compiles with pdfLaTeX, XeLaTeX and LuaLaTeX.
%
%% It may be distributed and/or modified under the
%% conditions of the LaTeX Project Public License, either version 1.3
%% of this license or (at your option) any later version.
%% The latest version of this license is in
%%    http://www.latex-project.org/lppl.txt
%% and version 1.3 or later is part of all distributions of LaTeX
%% version 2003/12/01 or later.
%%%%%%%%%%%%%%%%

%% Use the "normalphoto" option if you want a normal photo instead of cropped to a circle
% \documentclass[10pt,a4paper,normalphoto]{altacv}

\documentclass[10pt,letter,ragged2e,withhyper]{altacv}
%% AltaCV uses the fontawesome5 and packages.
%% See http://texdoc.net/pkg/fontawesome5 for full list of symbols.

% Change the page layout if you need to
\geometry{left=1.5cm,right=1.5cm,top=2cm,bottom=2cm,columnsep=1.cm}

% The paracol package lets you typeset columns of text in parallel
\usepackage{paracol}

% Change the font if you want to, depending on whether
% you're using pdflatex or xelatex/lualatex
\usepackage{hyperref}
\ifxetexorluatex
  % If using xelatex or lualatex:
  \setmainfont{Roboto Slab}
  \setsansfont{Lato}
  \renewcommand{\familydefault}{\sfdefault}
\else
  % If using pdflatex:
  \usepackage[rm]{roboto}
  \usepackage[defaultsans]{lato}
  % \usepackage{sourcesanspro}
  \renewcommand{\familydefault}{\sfdefault}
\fi

% Change the colours if you want to
\definecolor{SlateGrey}{HTML}{2E2E2E}
\definecolor{LightGrey}{HTML}{666666}
\definecolor{DarkPastelRed}{HTML}{003100}
\definecolor{PastelRed}{HTML}{4d7c4d}
\definecolor{GoldenEarth}{HTML}{023020}
%\definecolor{GoldenEarth}{HTML}{E7D192}
\colorlet{name}{black}
\colorlet{tagline}{PastelRed}
\colorlet{heading}{DarkPastelRed}
\colorlet{headingrule}{GoldenEarth}
\colorlet{subheading}{PastelRed}
\colorlet{accent}{PastelRed}
\colorlet{emphasis}{SlateGrey}
\colorlet{body}{LightGrey}

% Change some fonts, if necessary
\renewcommand{\namefont}{\Huge\rmfamily\bfseries}
\renewcommand{\personalinfofont}{\footnotesize}
\renewcommand{\cvsectionfont}{\LARGE\rmfamily\bfseries}
\renewcommand{\cvsubsectionfont}{\large\bfseries}


% Change the bullets for itemize and rating marker
% for \cvskill if you want to
\renewcommand{\itemmarker}{{\small\textbullet}}
\renewcommand{\ratingmarker}{\faCircle}

%% Use (and optionally edit if necessary) this .tex if you
%% want to use an author-year reference style like APA(6)
%% for your publication list
% \input{pubs-authoryear}

%% Use (and optionally edit if necessary) this .tex if you
%% want an originally numerical reference style like IEEE
%% for your publication list
\input{pubs-num}

%% sample.bib contains your publications
\addbibresource{sample.bib}

\begin{document}
\name{Sebastián Aguilera Novoa}
\tagline{Físico - Profesor - Programador de Python}
%% You can add multiple photos on the left or right
%\photoR{2.8cm}{Globe_High}
% \photoL{2.5cm}{Yacht_High,Suitcase_High}

\personalinfo{%
  % Not all of these are required!
  \email{saguileran@unal.edu.co} \hspace{-2pt}
  \phone{+57 3195140529}% \hspace{50pt}
  %\mailaddress{Åddrésş, Street, 00000 Cóuntry}
  \homepage{saguileran.github.io} \hspace{-2pt}%\hspace{50pt}
  %\twitter{@twitterhandle}
  \linkedin{saguileran} \hspace{-3pt}%\hspace{50pt}
  \github{saguileran} \hspace{5pt}
  \location{Bogotá D.C.}
  %\orcid{0000-0000-0000-0000}
  %% You can add your own arbitrary detail with
  %% \printinfo{symbol}{detail}[optional hyperlink prefix]
  % \printinfo{\faPaw}{Hey ho!}[https://example.com/]
  %% Or you can declare your own field with
  %% \NewInfoFiled{fieldname}{symbol}[optional hyperlink prefix] and use it:
  % \NewInfoField{gitlab}{\faGitlab}[https://gitlab.com/]
  % \gitlab{your_id}
  %%
  %% For services and platforms like Mastodon where there isn't a
  %% straightforward relation between the user ID/nickname and the hyperlink,
  %% you can use \printinfo directly e.g.
  % \printinfo{\faMastodon}{@username@instace}[https://instance.url/@username]
  %% But if you absolutely want to create new dedicated info fields for
  %% such platforms, then use \NewInfoField* with a star:
  % \NewInfoField*{mastodon}{\faMastodon}
  %% then you can use \mastodon, with TWO arguments where the 2nd argument is
  %% the full hyperlink.
  % \mastodon{@username@instance}{https://instance.url/@username}
}

\makecvheader
%% Depending on your tastes, you may want to make fonts of itemize environments slightly smaller
% \AtBeginEnvironment{itemize}{\small}

%% Set the left/right column width ratio to 6:4.
\columnratio{0.5}

% Start a 2-column paracol. Both the left and right columns will automatically
% break across pages if things get too long.
\begin{paracol}{2}
\cvsection{Experiencia}


\cvevent{Desarrollador Junior de Python}{AI GENERATIVE S.A.S. – AIGEN S.A.S}{Mar 2024 -- Presente}{remoto}
\begin{itemize}
\item Cree, despliegue y evalúe agentes inteligentes utilizando Python, la API ChatGPT y las API REST para generar contenidos digitales para los medios sociales. 
\end{itemize}

\divider

\cvevent{Desarrollador Junior de Python}{\href{www.marbai.ai}{MarBAI}}{Mar 2024 -- Presente}{remoto}
\begin{itemize}
\item Cree, despliegue y evalúe agentes inteligentes utilizando Python, la API ChatGPT y las API REST para generar contenidos digitales para los medios sociales.
\end{itemize}

\divider

\cvevent{\href{https://github.com/saguileran/ETITC-2024-1/}{Profesor Universitario}}{\href{https://www.etitc.edu.co/es/}{Escuela Tecnológica Instituto Técnico Central (ETITC)}}{Mar 2024 -- Nov 2024}{Bogotá, Colombia}
\begin{itemize}
\item Enseñanza de programación java (Backend y Frontend), bases de datos, sistemas operativos y estructura de datos a estudiantes universitarios de ingeniería de sistemas.
\end{itemize}

\divider

%\cvevent{Private Professor}{Freelance}{Jan 2016 -- Today}{Bogotá, Colombia}
%\begin{itemize}
%\item Teaching mathematics and physics to high school and university students.
%\end{itemize}

%\divider

%\cvevent{Customer Service Agent}{Concentrix}{Sep 2023 -- Dic 2023}{Bogotá, Colombia}
%\begin{itemize}
%\item Assist customers in placing orders and help them resolve order-related problems.
%\end{itemize}

%\divider


\cvevent{\href{https://saguileran.github.io/MD-SCPI/}{Internship in Molecular Modeling and Simulations}}{Instituto de Física de São Carlos - Universidad de São Paulo}{Feb 2023 -- Abr 2023}{São Carlos, Brazil}
\begin{itemize}
\item Configurar y contrastar simulaciones Monte Carlo (MC) y de dinámica molecular (MD) de un sistema proteína-ligando para generar datos estructurados (posiciones, velocidades, RMS, etc).
\item Analizar los datos estructurados generados por las simulaciones MC y MD mediante algoritmos de modelos de Markov (lagtime, cadenas de Markov ocultas, etc.).
\end{itemize}

\divider

\cvevent{\href{https://saguileran.github.io/maad/}{MAAD - Analisis de Paisajes Sonoros en Python}}{proyecto scikit-maad}{Ene 2023 -- Feb 2023}{Bogotá D.C., Colombia}
\begin{itemize}
\item Cree rasgos espectrales y temporales con ejemplos, pruebas y documentación. 
\end{itemize}


\divider

%\cvevent{\href{https://saguileran.github.io/birdsongs/}{Birdsongs}}{Research group Datalab - National University of Colombia}{2022}{Bogotá D.C., Colombia}
%\begin{itemize}
%\item Set up and evaluate a ML algorithm (using \href{https://github.com/HpWang-whu/YOHO}{YOHO/CNN}) using the data generated by the model as an augmentation data technique (\textit{in process}).
%\item Create and analyze synthetic birdsongs of the specie Zonotrichia Capensis from five countries.
%\item Generate unstructured data (audio/images of birdsongs) using a physical model and Python language.

%\end{itemize}

%\divider

\cvevent{\href{https://saguileran.github.io/assets/pdf/certificates/Delaware.PDF}{Programa de Investigación de Verano}}{Ingeniería Eléctrica e Informática - Universidad de Delaware}{Jun 2021 -- Sep 2021}{Delaware, EEUU}
\begin{itemize}
\item Analizar y visualizar datos de audio sin procesar de larga duración mediante procesamiento de señales y Matlab, registros continuos de 2 semanas de un micrófono en un río de la bahía de Delaware.
\end{itemize}



\cvsection{Proyectos}

\cvevent{\href{https://saguileran.github.io/birdsongs/}{Birdsongs}}{Universidad Nacional de Colombia}{Ago 2022 -- Presente}{Bogotá D.C., Colombia}
\begin{itemize}
\item Empaquetamiento en Python del modelo \textbf{gestos motores para el canto de los pájaros} que simula la producción de sonidos en aves.
\item Automatizar la generación de cantos de pájaros sintéticos (audio/imágenes) utilizando teoría y algoritmos de optimización numérica, métodos numéricos y procesamiento de señales.
\item Estudio y análisis del ancho de banda en función de la longitud de las sílabas trinadas (últimas sílabas de los cantos de los pájaros) para varios Zonotrichia Capensis de diferentes países.
\item Generar cantos de pájaros sintéticos comparables de algunas especies de aves colombianas: Zonotrichia Capensis, Rhinocryptidae y Mimus Gilvus.
%\item Generate and evaluate audio/images of synthetic birdsongs, unstructured data.

\end{itemize}
%A short abstract would also work.

\divider

\cvevent{\href{https://saguileran.github.io/MD-SCPI/}{Modelizado y Simulación de Moléculas}}{Universidad de Sao Paulo}{Feb 2023 -- Abr 2023}{Sao Carlos, Brazil}
\begin{itemize}
\item Estudio y evaluación de simulaciones moleculares de la cinética de des/ligadura en un sistema proteína-ligando.
\item Puesta a punto y ejecución de varias simulaciones MD y MC para diferentes sistemas.
\item Análisis de los eventos de desligamiento/ligamiento de las simulaciones MD y MC mediante análisis numérico.
\end{itemize}
%A short abstract would also work.

\divider

\cvevent{\href{https://saguileran.github.io/aprender/}{Aprender - Una Nueva Forma de Aprender}}{Independiente}{2019 -- 2022}{Bogotá D.C., Colombia}
\begin{itemize}
\item Diseñar, crear y alojar una página web para el preparatorio.
\item Implantar la plataforma Moodle en la página de inicio como plataforma de gestión del aprendizaje.
\item Profesor de matemáticas y física: diseño y creación de lecciones y pruebas de evaluación.
% with interactive activities using virtual tools such as PhET simulations, and tests for evaluation.
\end{itemize}
%A short abstract would also work.

\divider

\cvevent{\href{https://github.com/saguileran/Acoustics-Instruments}{Caracterización de una Flauta}}{Universidad Nacional de Colombia}{2020}{Bogotá D.C., Colombia}
\begin{itemize}
\item Estudiar el instrumento musical flauta dulce desde la física experimental, teórica y computacional.
\item Analizar y visualizar los datos estructurados generados y medidos a partir del estudio para compararlos.
\item Visualización de la presión acústica de las ondas de desarrollo mediante un LBM y Paraview para su comparación con las mediciones.

\end{itemize}
%A short abstract would also work.

\divider

\cvevent{\href{https://github.com/danielfelipe-df/Metodos-de-simulacion}{Simulación Acústica de un Aula de Clases}}{Universidad Nacional de Colombia}{2019}{Bogotá D.C., Colombia}
\begin{itemize}
\item Modelado y simulación de un aula de conferencias utilizando el método Lattice Boltzmann (LBM), escribiendo en c++ utilizando OOP, para generar datos estructurados comparables.
\item Medición física y computacional del tiempo de reverberación de las aulas para comparar.
\end{itemize}
%A short abstract would also work.

\divider

\cvevent{Laboratorios de Física}{Universidad Nacional de Colombia}{Ago 2015 -- Dic 2021}{Bogotá D.C., Colombia}
\begin{itemize}
\item Crear laboratorios para validar las teorías físicas mediante la medición de datos estructurados (magnitudes físicas mensurables).
\item Elabore informes de laboratorio con el estado de la cuestión, la discusión y el análisis (que incluya ajustes matemáticos a los datos estructurados), la metodología y las conclusiones.
\end{itemize}
%A short abstract would also work.


%\medskip

 %\cvsection{A Day of My Life}

% % Adapted from @Jake's answer from http://tex.stackexchange.com/a/82729/226
  %\wheelchart{outer radius}{inner radius}{
  %comma-separated list of value/text width/color/detail}
 %\wheelchart{1.5cm}{0.5cm}{%
   %6/8em/accent!30/{Sleep,\\beautiful sleep},
   %3/8em/accent!40/Hopeful novelist by night,
   %8/8em/accent!60/Daytime job,
   %2/10em/accent/Sports and relaxation,
   %5/6em/accent!20/Spending time with family
 %}

% % use ONLY \newpage if you want to force a page break for
% % ONLY the current column
% \newpage




\smallskip

\switchcolumn

% \cvsection{My Life Philosophy}

% \begin{quote}
% ``Something smart or heartfelt, preferably in one sentence.''
% \end{quote}

\cvsection{interesado en}

\smallskip

\cvtag{Acústica}
\cvtag{Aprendizaje de Máquina}
\cvtag{Idiomas}

\smallskip

% \cvtag{Bioacustics}
\cvtag{Ecología}
\cvtag{Simulaciones}
\cvtag{Música}
\cvtag{Pedagogía}



\cvsection{Educación}

% \cvevent{Ph.D.\ in Your Discipline}{Your University}{Sept 2002 -- June 2006}{}
% Thesis title: Wonderful Research

% \divider

% \cvevent{M.Sc.\ in Your Discipline}{Your University}{Sept 2001 -- June 2002}{}

\cvevent{Pregrado en Física}{Universidad Nacional de Colombia}{2015 -- 2023}{Bogotá D.C.}

\divider

\cvevent{Técnico en Instalaciones Eléctricas Residenciales}{Servicio Nacional de Aprendizaje (SENA)}{2012 -- 2013}{Bogotá D.C.}

\cvsection{Más orgulloso de}

\cvachievement{\faGraduationCap}{\href{https://saguileran.github.io/assets/pdf/certificates/USP.pdf}{Awarded Jhoti and Salazar Scholarship}}{Instituto de Física de São Carlos, USP}{2023}{São Carlos, Brazil}

\divider

\cvachievement{\faTrophy}{Segundo mejor trabajo de grado en física}{Departamento de Física, UNAL}{2023}{Bogotá, Colombia}

%\divider

%\cvachievement{\faHeartbeat}{Another achievement}{more details about it of course}

\cvsection{Fortalezas}

\smallskip

\cvtag{Trabajador}
\cvtag{Detallista}
\cvtag{Aprendizaje Rápido}

\smallskip

\cvtag{Creatividad}
\cvtag{Adaptabilidad}
\cvtag{Apasionado}
\cvtag{Curioso}

\smallskip \divider

\cvtag{Solución de Problemas}
\cvtag{Autodidacta}
\cvtag{Liderazgo}

\smallskip

\cvtag{Modelamiento Matemático}
\cvtag{Pensamiento Crítico}


\divider

\cvtag{Escucha activa}
\cvtag{Empatía}
\cvtag{Paciencia}
\cvtag{Análisis}
%\smallskip \divider
%
%\cvtag{Numerical Optimization}
%\cvtag{Signal Processing}
%
%\smallskip
%
%\cvtag{Mathematical Modeling}
%\cvtag{Critical Analysis}

%\cvtag{Embedded Systems}\\
%\cvtag{Statistical Analysis}

\vspace{50pt}

\cvsection{Idiomas}
\cvskill{Español}{5}
\cvskill{Inglés}{4.5}
\cvskill{Portugues}{3}
\cvskill{Alemán}{0.5}
%\divider

\vspace{10pt}

%\cvskill{German}{3.5} %% Supports X.5 values.

%% Yeah I didn't spend too much time making all the
%% spacing consistent... sorry. Use \smallskip, \medskip,
%% \bigskip, \vspace etc to make adjustments.
\medskip
%\medskip

%\vspace{50pt}
%\vfill


% \divider

\cvsection{Talleres/Eventos}

\cvevent{\href{https://saguileran.github.io/assets/pdf/certificates/MAPI3.pdf}{Presentación de Poster - III Conferencia Colombiana de Matemáticas Aplicadas e Industriales (MAPI 3)}}{Comisión de Matemáticas Aplicadas e Industriales de la Sociedad Colombiana de Matemáticas}{Junio, 2024}{Bucaramanga, Colombia}

\cvevent{Aprendizaje de Máquina Para la Materia y la Tecnología Cuántica}{Workshop - Organizado por Universidad de los Andes }{Mayo, 2019}{Bogotá, Colombia}

\vspace{10pt}

\cvsection{Publicaciones}

%% Specify your last name(s) and first name(s) as given in the .bib to automatically bold your own name in the publications list. 
%% One caveat: You need to write \bibnamedelima where there's a space in your name for this to work properly; or write \bibnamedelimi if you use initials in the .bib
%% You can specify multiple names, especially if you have changed your name or if you need to highlight multiple authors. 
\mynames{Lim/Lian\bibnamedelima Tze,
  Wong/Lian\bibnamedelima Tze,
  Lim/Tracy,
  Lim/L.\bibnamedelimi T.}
%% MAKE SURE THERE IS NO SPACE AFTER THE FINAL NAME IN YOUR \mynames LIST

\nocite{*}

\printbibliography[heading=pubtype,title={\printinfo{\faBook}{Books}},type=book]

% \divider


% \printbibliography[heading=pubtype,title={\printinfo{\faFile*[regular]}{Journal Articles}},type=article]

% \divider

% \printbibliography[heading=pubtype,title={\printinfo{\faUsers}{Conference Proceedings}},type=inproceedings]

%% Switch to the right column. This will now automatically move to the second
%% page if the content is too long.

\vspace{10pt}
\cvsection{habilidades informáticas}

\cvskill{Python, Latex, Office, VSC}{4.5}
\cvskill{Java, Github, Linux, SSH, Matlab}{4}
\cvskill{Jupyter-Notebook, Markdown}{4}
\cvskill{Julia,  Krita, C++, Power BI}{3.5}
\cvskill{Mathematica, Canva}{3.5}
\cvskill{JS, HTML, SQL, CSS, Workbrench}{3}

\smallskip
\vspace{10pt}
\cvsection{Librerias de Python}

\smallskip

\cvskill{Matplotlib, Numpy, Plotly, Pandas}{4.5}
\cvskill{Tensorflow, OpenAI, Scipy}{3.5}
\cvskill{Pytorch, Scikit-Learn, Pytest, Lmfit}{3}
\cvskill{PeakUtils, Sympy, Seaborn, Librosa}{3}
\cvskill{Scrapy, Django, Flask, OpenCV}{1}


\vspace{100pt}
\cvsection{Formación/Certificaciones}

\cvevent{\href{https://www.coursera.org/account/accomplishments/verify/EEAPCA6DUALG}{Modelos Secuenciales}}{Coursera}{2024}{En Línea}

\cvevent{\href{https://www.coursera.org/account/accomplishments/verify/WJKJG4RKL7NR}{Redes Neuronales Convolucionales}}{Coursera}{2023}{En Línea}

\cvevent{\href{https://www.coursera.org/account/accomplishments/verify/H76ELSAQH37Y}{Estructuración de Proyectos de Aprendizaje de Máquina}}{Coursera}{2023}{En Línea}


\cvevent{\href{https://www.coursera.org/account/accomplishments/verify/Z2M58E79QSMM}{Introducción al Lenguaje de Consulta Estructurado (SQL)}}{Coursera}{2022}{En Línea}


\cvevent{\href{https://www.coursera.org/account/accomplishments/verify/JKPQ8LM8KWUC}{Redes Neuronales y Aprendizaje Profundo}}{Coursera}{2021}{En Línea}

\cvevent{\href{https://www.coursera.org/account/accomplishments/verify/9WF6F7FQNH38}{Mejora de las Redes Neuronales Profundas: Ajuste de Hiperparámetros, Regularización y Optimización}}{Coursera}{2021}{En Línea}


\cvsection{Referencias}

% \cvref{name}{email}{mailing address}
%\cvref{Prof.\ Alpha Beta}{Institute}{a.beta@university.edu} {Address Line 1\\Address line 2}

\cvref{Prof.\ Francisco Gómez Jaramillo}{Universidad Nacional de Colombia (UNAL)}
{fagomezj@unal.edu.co}
\hspace{0.05pt}
%\divider


\cvref{Prof.\ Gabo Mindlin}{Universidad de Buenos Aires (UBA), Argentina}
{gabo@df.uba.ar}
\hspace{0.05pt}

%\cvref{Res.\ Juan Sebastián  Ulloa}{Instituto de Investigación de Recursos Biológicos
%Alexander von Humboldt, CO}
%{julloa@humboldt.org.co}
%\hspace{0.05pt}

\cvref{Prof.\ Alessandro S. Nascimento}{Universidad de São Paulo (USP), Brazil}
{asnascimento@ifsc.usp.br}


% \divider


\end{paracol}

\end{document}
